
% Default to the notebook output style

    


% Inherit from the specified cell style.




    
\documentclass[11pt]{article}

    
    
    \usepackage[T1]{fontenc}
    % Nicer default font (+ math font) than Computer Modern for most use cases
    \usepackage{mathpazo}

    % Basic figure setup, for now with no caption control since it's done
    % automatically by Pandoc (which extracts ![](path) syntax from Markdown).
    \usepackage{graphicx}
    % We will generate all images so they have a width \maxwidth. This means
    % that they will get their normal width if they fit onto the page, but
    % are scaled down if they would overflow the margins.
    \makeatletter
    \def\maxwidth{\ifdim\Gin@nat@width>\linewidth\linewidth
    \else\Gin@nat@width\fi}
    \makeatother
    \let\Oldincludegraphics\includegraphics
    % Set max figure width to be 80% of text width, for now hardcoded.
    \renewcommand{\includegraphics}[1]{\Oldincludegraphics[width=.8\maxwidth]{#1}}
    % Ensure that by default, figures have no caption (until we provide a
    % proper Figure object with a Caption API and a way to capture that
    % in the conversion process - todo).
    \usepackage{caption}
    \DeclareCaptionLabelFormat{nolabel}{}
    \captionsetup{labelformat=nolabel}

    \usepackage{adjustbox} % Used to constrain images to a maximum size 
    \usepackage{xcolor} % Allow colors to be defined
    \usepackage{enumerate} % Needed for markdown enumerations to work
    \usepackage{geometry} % Used to adjust the document margins
    \usepackage{amsmath} % Equations
    \usepackage{amssymb} % Equations
    \usepackage{textcomp} % defines textquotesingle
    % Hack from http://tex.stackexchange.com/a/47451/13684:
    \AtBeginDocument{%
        \def\PYZsq{\textquotesingle}% Upright quotes in Pygmentized code
    }
    \usepackage{upquote} % Upright quotes for verbatim code
    \usepackage{eurosym} % defines \euro
    \usepackage[mathletters]{ucs} % Extended unicode (utf-8) support
    \usepackage[utf8x]{inputenc} % Allow utf-8 characters in the tex document
    \usepackage{fancyvrb} % verbatim replacement that allows latex
    \usepackage{grffile} % extends the file name processing of package graphics 
                         % to support a larger range 
    % The hyperref package gives us a pdf with properly built
    % internal navigation ('pdf bookmarks' for the table of contents,
    % internal cross-reference links, web links for URLs, etc.)
    \usepackage{hyperref}
    \usepackage{longtable} % longtable support required by pandoc >1.10
    \usepackage{booktabs}  % table support for pandoc > 1.12.2
    \usepackage[inline]{enumitem} % IRkernel/repr support (it uses the enumerate* environment)
    \usepackage[normalem]{ulem} % ulem is needed to support strikethroughs (\sout)
                                % normalem makes italics be italics, not underlines
    

    
    
    % Colors for the hyperref package
    \definecolor{urlcolor}{rgb}{0,.145,.698}
    \definecolor{linkcolor}{rgb}{.71,0.21,0.01}
    \definecolor{citecolor}{rgb}{.12,.54,.11}

    % ANSI colors
    \definecolor{ansi-black}{HTML}{3E424D}
    \definecolor{ansi-black-intense}{HTML}{282C36}
    \definecolor{ansi-red}{HTML}{E75C58}
    \definecolor{ansi-red-intense}{HTML}{B22B31}
    \definecolor{ansi-green}{HTML}{00A250}
    \definecolor{ansi-green-intense}{HTML}{007427}
    \definecolor{ansi-yellow}{HTML}{DDB62B}
    \definecolor{ansi-yellow-intense}{HTML}{B27D12}
    \definecolor{ansi-blue}{HTML}{208FFB}
    \definecolor{ansi-blue-intense}{HTML}{0065CA}
    \definecolor{ansi-magenta}{HTML}{D160C4}
    \definecolor{ansi-magenta-intense}{HTML}{A03196}
    \definecolor{ansi-cyan}{HTML}{60C6C8}
    \definecolor{ansi-cyan-intense}{HTML}{258F8F}
    \definecolor{ansi-white}{HTML}{C5C1B4}
    \definecolor{ansi-white-intense}{HTML}{A1A6B2}

    % commands and environments needed by pandoc snippets
    % extracted from the output of `pandoc -s`
    \providecommand{\tightlist}{%
      \setlength{\itemsep}{0pt}\setlength{\parskip}{0pt}}
    \DefineVerbatimEnvironment{Highlighting}{Verbatim}{commandchars=\\\{\}}
    % Add ',fontsize=\small' for more characters per line
    \newenvironment{Shaded}{}{}
    \newcommand{\KeywordTok}[1]{\textcolor[rgb]{0.00,0.44,0.13}{\textbf{{#1}}}}
    \newcommand{\DataTypeTok}[1]{\textcolor[rgb]{0.56,0.13,0.00}{{#1}}}
    \newcommand{\DecValTok}[1]{\textcolor[rgb]{0.25,0.63,0.44}{{#1}}}
    \newcommand{\BaseNTok}[1]{\textcolor[rgb]{0.25,0.63,0.44}{{#1}}}
    \newcommand{\FloatTok}[1]{\textcolor[rgb]{0.25,0.63,0.44}{{#1}}}
    \newcommand{\CharTok}[1]{\textcolor[rgb]{0.25,0.44,0.63}{{#1}}}
    \newcommand{\StringTok}[1]{\textcolor[rgb]{0.25,0.44,0.63}{{#1}}}
    \newcommand{\CommentTok}[1]{\textcolor[rgb]{0.38,0.63,0.69}{\textit{{#1}}}}
    \newcommand{\OtherTok}[1]{\textcolor[rgb]{0.00,0.44,0.13}{{#1}}}
    \newcommand{\AlertTok}[1]{\textcolor[rgb]{1.00,0.00,0.00}{\textbf{{#1}}}}
    \newcommand{\FunctionTok}[1]{\textcolor[rgb]{0.02,0.16,0.49}{{#1}}}
    \newcommand{\RegionMarkerTok}[1]{{#1}}
    \newcommand{\ErrorTok}[1]{\textcolor[rgb]{1.00,0.00,0.00}{\textbf{{#1}}}}
    \newcommand{\NormalTok}[1]{{#1}}
    
    % Additional commands for more recent versions of Pandoc
    \newcommand{\ConstantTok}[1]{\textcolor[rgb]{0.53,0.00,0.00}{{#1}}}
    \newcommand{\SpecialCharTok}[1]{\textcolor[rgb]{0.25,0.44,0.63}{{#1}}}
    \newcommand{\VerbatimStringTok}[1]{\textcolor[rgb]{0.25,0.44,0.63}{{#1}}}
    \newcommand{\SpecialStringTok}[1]{\textcolor[rgb]{0.73,0.40,0.53}{{#1}}}
    \newcommand{\ImportTok}[1]{{#1}}
    \newcommand{\DocumentationTok}[1]{\textcolor[rgb]{0.73,0.13,0.13}{\textit{{#1}}}}
    \newcommand{\AnnotationTok}[1]{\textcolor[rgb]{0.38,0.63,0.69}{\textbf{\textit{{#1}}}}}
    \newcommand{\CommentVarTok}[1]{\textcolor[rgb]{0.38,0.63,0.69}{\textbf{\textit{{#1}}}}}
    \newcommand{\VariableTok}[1]{\textcolor[rgb]{0.10,0.09,0.49}{{#1}}}
    \newcommand{\ControlFlowTok}[1]{\textcolor[rgb]{0.00,0.44,0.13}{\textbf{{#1}}}}
    \newcommand{\OperatorTok}[1]{\textcolor[rgb]{0.40,0.40,0.40}{{#1}}}
    \newcommand{\BuiltInTok}[1]{{#1}}
    \newcommand{\ExtensionTok}[1]{{#1}}
    \newcommand{\PreprocessorTok}[1]{\textcolor[rgb]{0.74,0.48,0.00}{{#1}}}
    \newcommand{\AttributeTok}[1]{\textcolor[rgb]{0.49,0.56,0.16}{{#1}}}
    \newcommand{\InformationTok}[1]{\textcolor[rgb]{0.38,0.63,0.69}{\textbf{\textit{{#1}}}}}
    \newcommand{\WarningTok}[1]{\textcolor[rgb]{0.38,0.63,0.69}{\textbf{\textit{{#1}}}}}
    
    
    % Define a nice break command that doesn't care if a line doesn't already
    % exist.
    \def\br{\hspace*{\fill} \\* }
    % Math Jax compatability definitions
    \def\gt{>}
    \def\lt{<}
    % Document parameters
    \title{03.02-matplotlib-basics}
    
    
    

    % Pygments definitions
    
\makeatletter
\def\PY@reset{\let\PY@it=\relax \let\PY@bf=\relax%
    \let\PY@ul=\relax \let\PY@tc=\relax%
    \let\PY@bc=\relax \let\PY@ff=\relax}
\def\PY@tok#1{\csname PY@tok@#1\endcsname}
\def\PY@toks#1+{\ifx\relax#1\empty\else%
    \PY@tok{#1}\expandafter\PY@toks\fi}
\def\PY@do#1{\PY@bc{\PY@tc{\PY@ul{%
    \PY@it{\PY@bf{\PY@ff{#1}}}}}}}
\def\PY#1#2{\PY@reset\PY@toks#1+\relax+\PY@do{#2}}

\expandafter\def\csname PY@tok@w\endcsname{\def\PY@tc##1{\textcolor[rgb]{0.73,0.73,0.73}{##1}}}
\expandafter\def\csname PY@tok@c\endcsname{\let\PY@it=\textit\def\PY@tc##1{\textcolor[rgb]{0.25,0.50,0.50}{##1}}}
\expandafter\def\csname PY@tok@cp\endcsname{\def\PY@tc##1{\textcolor[rgb]{0.74,0.48,0.00}{##1}}}
\expandafter\def\csname PY@tok@k\endcsname{\let\PY@bf=\textbf\def\PY@tc##1{\textcolor[rgb]{0.00,0.50,0.00}{##1}}}
\expandafter\def\csname PY@tok@kp\endcsname{\def\PY@tc##1{\textcolor[rgb]{0.00,0.50,0.00}{##1}}}
\expandafter\def\csname PY@tok@kt\endcsname{\def\PY@tc##1{\textcolor[rgb]{0.69,0.00,0.25}{##1}}}
\expandafter\def\csname PY@tok@o\endcsname{\def\PY@tc##1{\textcolor[rgb]{0.40,0.40,0.40}{##1}}}
\expandafter\def\csname PY@tok@ow\endcsname{\let\PY@bf=\textbf\def\PY@tc##1{\textcolor[rgb]{0.67,0.13,1.00}{##1}}}
\expandafter\def\csname PY@tok@nb\endcsname{\def\PY@tc##1{\textcolor[rgb]{0.00,0.50,0.00}{##1}}}
\expandafter\def\csname PY@tok@nf\endcsname{\def\PY@tc##1{\textcolor[rgb]{0.00,0.00,1.00}{##1}}}
\expandafter\def\csname PY@tok@nc\endcsname{\let\PY@bf=\textbf\def\PY@tc##1{\textcolor[rgb]{0.00,0.00,1.00}{##1}}}
\expandafter\def\csname PY@tok@nn\endcsname{\let\PY@bf=\textbf\def\PY@tc##1{\textcolor[rgb]{0.00,0.00,1.00}{##1}}}
\expandafter\def\csname PY@tok@ne\endcsname{\let\PY@bf=\textbf\def\PY@tc##1{\textcolor[rgb]{0.82,0.25,0.23}{##1}}}
\expandafter\def\csname PY@tok@nv\endcsname{\def\PY@tc##1{\textcolor[rgb]{0.10,0.09,0.49}{##1}}}
\expandafter\def\csname PY@tok@no\endcsname{\def\PY@tc##1{\textcolor[rgb]{0.53,0.00,0.00}{##1}}}
\expandafter\def\csname PY@tok@nl\endcsname{\def\PY@tc##1{\textcolor[rgb]{0.63,0.63,0.00}{##1}}}
\expandafter\def\csname PY@tok@ni\endcsname{\let\PY@bf=\textbf\def\PY@tc##1{\textcolor[rgb]{0.60,0.60,0.60}{##1}}}
\expandafter\def\csname PY@tok@na\endcsname{\def\PY@tc##1{\textcolor[rgb]{0.49,0.56,0.16}{##1}}}
\expandafter\def\csname PY@tok@nt\endcsname{\let\PY@bf=\textbf\def\PY@tc##1{\textcolor[rgb]{0.00,0.50,0.00}{##1}}}
\expandafter\def\csname PY@tok@nd\endcsname{\def\PY@tc##1{\textcolor[rgb]{0.67,0.13,1.00}{##1}}}
\expandafter\def\csname PY@tok@s\endcsname{\def\PY@tc##1{\textcolor[rgb]{0.73,0.13,0.13}{##1}}}
\expandafter\def\csname PY@tok@sd\endcsname{\let\PY@it=\textit\def\PY@tc##1{\textcolor[rgb]{0.73,0.13,0.13}{##1}}}
\expandafter\def\csname PY@tok@si\endcsname{\let\PY@bf=\textbf\def\PY@tc##1{\textcolor[rgb]{0.73,0.40,0.53}{##1}}}
\expandafter\def\csname PY@tok@se\endcsname{\let\PY@bf=\textbf\def\PY@tc##1{\textcolor[rgb]{0.73,0.40,0.13}{##1}}}
\expandafter\def\csname PY@tok@sr\endcsname{\def\PY@tc##1{\textcolor[rgb]{0.73,0.40,0.53}{##1}}}
\expandafter\def\csname PY@tok@ss\endcsname{\def\PY@tc##1{\textcolor[rgb]{0.10,0.09,0.49}{##1}}}
\expandafter\def\csname PY@tok@sx\endcsname{\def\PY@tc##1{\textcolor[rgb]{0.00,0.50,0.00}{##1}}}
\expandafter\def\csname PY@tok@m\endcsname{\def\PY@tc##1{\textcolor[rgb]{0.40,0.40,0.40}{##1}}}
\expandafter\def\csname PY@tok@gh\endcsname{\let\PY@bf=\textbf\def\PY@tc##1{\textcolor[rgb]{0.00,0.00,0.50}{##1}}}
\expandafter\def\csname PY@tok@gu\endcsname{\let\PY@bf=\textbf\def\PY@tc##1{\textcolor[rgb]{0.50,0.00,0.50}{##1}}}
\expandafter\def\csname PY@tok@gd\endcsname{\def\PY@tc##1{\textcolor[rgb]{0.63,0.00,0.00}{##1}}}
\expandafter\def\csname PY@tok@gi\endcsname{\def\PY@tc##1{\textcolor[rgb]{0.00,0.63,0.00}{##1}}}
\expandafter\def\csname PY@tok@gr\endcsname{\def\PY@tc##1{\textcolor[rgb]{1.00,0.00,0.00}{##1}}}
\expandafter\def\csname PY@tok@ge\endcsname{\let\PY@it=\textit}
\expandafter\def\csname PY@tok@gs\endcsname{\let\PY@bf=\textbf}
\expandafter\def\csname PY@tok@gp\endcsname{\let\PY@bf=\textbf\def\PY@tc##1{\textcolor[rgb]{0.00,0.00,0.50}{##1}}}
\expandafter\def\csname PY@tok@go\endcsname{\def\PY@tc##1{\textcolor[rgb]{0.53,0.53,0.53}{##1}}}
\expandafter\def\csname PY@tok@gt\endcsname{\def\PY@tc##1{\textcolor[rgb]{0.00,0.27,0.87}{##1}}}
\expandafter\def\csname PY@tok@err\endcsname{\def\PY@bc##1{\setlength{\fboxsep}{0pt}\fcolorbox[rgb]{1.00,0.00,0.00}{1,1,1}{\strut ##1}}}
\expandafter\def\csname PY@tok@kc\endcsname{\let\PY@bf=\textbf\def\PY@tc##1{\textcolor[rgb]{0.00,0.50,0.00}{##1}}}
\expandafter\def\csname PY@tok@kd\endcsname{\let\PY@bf=\textbf\def\PY@tc##1{\textcolor[rgb]{0.00,0.50,0.00}{##1}}}
\expandafter\def\csname PY@tok@kn\endcsname{\let\PY@bf=\textbf\def\PY@tc##1{\textcolor[rgb]{0.00,0.50,0.00}{##1}}}
\expandafter\def\csname PY@tok@kr\endcsname{\let\PY@bf=\textbf\def\PY@tc##1{\textcolor[rgb]{0.00,0.50,0.00}{##1}}}
\expandafter\def\csname PY@tok@bp\endcsname{\def\PY@tc##1{\textcolor[rgb]{0.00,0.50,0.00}{##1}}}
\expandafter\def\csname PY@tok@fm\endcsname{\def\PY@tc##1{\textcolor[rgb]{0.00,0.00,1.00}{##1}}}
\expandafter\def\csname PY@tok@vc\endcsname{\def\PY@tc##1{\textcolor[rgb]{0.10,0.09,0.49}{##1}}}
\expandafter\def\csname PY@tok@vg\endcsname{\def\PY@tc##1{\textcolor[rgb]{0.10,0.09,0.49}{##1}}}
\expandafter\def\csname PY@tok@vi\endcsname{\def\PY@tc##1{\textcolor[rgb]{0.10,0.09,0.49}{##1}}}
\expandafter\def\csname PY@tok@vm\endcsname{\def\PY@tc##1{\textcolor[rgb]{0.10,0.09,0.49}{##1}}}
\expandafter\def\csname PY@tok@sa\endcsname{\def\PY@tc##1{\textcolor[rgb]{0.73,0.13,0.13}{##1}}}
\expandafter\def\csname PY@tok@sb\endcsname{\def\PY@tc##1{\textcolor[rgb]{0.73,0.13,0.13}{##1}}}
\expandafter\def\csname PY@tok@sc\endcsname{\def\PY@tc##1{\textcolor[rgb]{0.73,0.13,0.13}{##1}}}
\expandafter\def\csname PY@tok@dl\endcsname{\def\PY@tc##1{\textcolor[rgb]{0.73,0.13,0.13}{##1}}}
\expandafter\def\csname PY@tok@s2\endcsname{\def\PY@tc##1{\textcolor[rgb]{0.73,0.13,0.13}{##1}}}
\expandafter\def\csname PY@tok@sh\endcsname{\def\PY@tc##1{\textcolor[rgb]{0.73,0.13,0.13}{##1}}}
\expandafter\def\csname PY@tok@s1\endcsname{\def\PY@tc##1{\textcolor[rgb]{0.73,0.13,0.13}{##1}}}
\expandafter\def\csname PY@tok@mb\endcsname{\def\PY@tc##1{\textcolor[rgb]{0.40,0.40,0.40}{##1}}}
\expandafter\def\csname PY@tok@mf\endcsname{\def\PY@tc##1{\textcolor[rgb]{0.40,0.40,0.40}{##1}}}
\expandafter\def\csname PY@tok@mh\endcsname{\def\PY@tc##1{\textcolor[rgb]{0.40,0.40,0.40}{##1}}}
\expandafter\def\csname PY@tok@mi\endcsname{\def\PY@tc##1{\textcolor[rgb]{0.40,0.40,0.40}{##1}}}
\expandafter\def\csname PY@tok@il\endcsname{\def\PY@tc##1{\textcolor[rgb]{0.40,0.40,0.40}{##1}}}
\expandafter\def\csname PY@tok@mo\endcsname{\def\PY@tc##1{\textcolor[rgb]{0.40,0.40,0.40}{##1}}}
\expandafter\def\csname PY@tok@ch\endcsname{\let\PY@it=\textit\def\PY@tc##1{\textcolor[rgb]{0.25,0.50,0.50}{##1}}}
\expandafter\def\csname PY@tok@cm\endcsname{\let\PY@it=\textit\def\PY@tc##1{\textcolor[rgb]{0.25,0.50,0.50}{##1}}}
\expandafter\def\csname PY@tok@cpf\endcsname{\let\PY@it=\textit\def\PY@tc##1{\textcolor[rgb]{0.25,0.50,0.50}{##1}}}
\expandafter\def\csname PY@tok@c1\endcsname{\let\PY@it=\textit\def\PY@tc##1{\textcolor[rgb]{0.25,0.50,0.50}{##1}}}
\expandafter\def\csname PY@tok@cs\endcsname{\let\PY@it=\textit\def\PY@tc##1{\textcolor[rgb]{0.25,0.50,0.50}{##1}}}

\def\PYZbs{\char`\\}
\def\PYZus{\char`\_}
\def\PYZob{\char`\{}
\def\PYZcb{\char`\}}
\def\PYZca{\char`\^}
\def\PYZam{\char`\&}
\def\PYZlt{\char`\<}
\def\PYZgt{\char`\>}
\def\PYZsh{\char`\#}
\def\PYZpc{\char`\%}
\def\PYZdl{\char`\$}
\def\PYZhy{\char`\-}
\def\PYZsq{\char`\'}
\def\PYZdq{\char`\"}
\def\PYZti{\char`\~}
% for compatibility with earlier versions
\def\PYZat{@}
\def\PYZlb{[}
\def\PYZrb{]}
\makeatother


    % Exact colors from NB
    \definecolor{incolor}{rgb}{0.0, 0.0, 0.5}
    \definecolor{outcolor}{rgb}{0.545, 0.0, 0.0}



    
    % Prevent overflowing lines due to hard-to-break entities
    \sloppy 
    % Setup hyperref package
    \hypersetup{
      breaklinks=true,  % so long urls are correctly broken across lines
      colorlinks=true,
      urlcolor=urlcolor,
      linkcolor=linkcolor,
      citecolor=citecolor,
      }
    % Slightly bigger margins than the latex defaults
    
    \geometry{verbose,tmargin=1in,bmargin=1in,lmargin=1in,rmargin=1in}
    
    

    \begin{document}
    
    
    \maketitle
    
    

    
    \section{Matplotlib 基础}\label{matplotlib-ux57faux7840}

    在使用\textbf{Numpy}之前,需要了解一些画图的基础。

\textbf{Matplotlib}是一个类似\textbf{Matlab}的工具包,主页地址为

http://matplotlib.org

导入 \texttt{matplotlib} 和 \texttt{numpy}:

    \begin{Verbatim}[commandchars=\\\{\}]
{\color{incolor}In [{\color{incolor}1}]:} \PY{o}{\PYZpc{}}\PY{k}{pylab}
\end{Verbatim}


    \begin{Verbatim}[commandchars=\\\{\}]
Using matplotlib backend: Qt4Agg
Populating the interactive namespace from numpy and matplotlib

    \end{Verbatim}

    \subsection{plot 二维图}\label{plot-ux4e8cux7ef4ux56fe}

    \begin{Shaded}
\begin{Highlighting}[]
\NormalTok{plot(y)}
\NormalTok{plot(x, y)}
\NormalTok{plot(x, y, format_string)}
\end{Highlighting}
\end{Shaded}

只给定 \texttt{y} 值,默认以下标为 \texttt{x} 轴:

    \begin{Verbatim}[commandchars=\\\{\}]
{\color{incolor}In [{\color{incolor}2}]:} \PY{o}{\PYZpc{}}\PY{k}{matplotlib} inline
        \PY{n}{x} \PY{o}{=} \PY{n}{linspace}\PY{p}{(}\PY{l+m+mi}{0}\PY{p}{,} \PY{l+m+mi}{2} \PY{o}{*} \PY{n}{pi}\PY{p}{,} \PY{l+m+mi}{50}\PY{p}{)}
        \PY{n}{plot}\PY{p}{(}\PY{n}{sin}\PY{p}{(}\PY{n}{x}\PY{p}{)}\PY{p}{)}
\end{Verbatim}


\begin{Verbatim}[commandchars=\\\{\}]
{\color{outcolor}Out[{\color{outcolor}2}]:} [<matplotlib.lines.Line2D at 0xa086fd0>]
\end{Verbatim}
            
    \begin{center}
    \adjustimage{max size={0.9\linewidth}{0.9\paperheight}}{output_5_1.png}
    \end{center}
    { \hspace*{\fill} \\}
    
    给定 \texttt{x} 和 \texttt{y} 值:

    \begin{Verbatim}[commandchars=\\\{\}]
{\color{incolor}In [{\color{incolor}3}]:} \PY{n}{plot}\PY{p}{(}\PY{n}{x}\PY{p}{,} \PY{n}{sin}\PY{p}{(}\PY{n}{x}\PY{p}{)}\PY{p}{)}
\end{Verbatim}


\begin{Verbatim}[commandchars=\\\{\}]
{\color{outcolor}Out[{\color{outcolor}3}]:} [<matplotlib.lines.Line2D at 0xa241898>]
\end{Verbatim}
            
    \begin{center}
    \adjustimage{max size={0.9\linewidth}{0.9\paperheight}}{output_7_1.png}
    \end{center}
    { \hspace*{\fill} \\}
    
    多条数据线:

    \begin{Verbatim}[commandchars=\\\{\}]
{\color{incolor}In [{\color{incolor}4}]:} \PY{n}{plot}\PY{p}{(}\PY{n}{x}\PY{p}{,} \PY{n}{sin}\PY{p}{(}\PY{n}{x}\PY{p}{)}\PY{p}{,}
            \PY{n}{x}\PY{p}{,} \PY{n}{sin}\PY{p}{(}\PY{l+m+mi}{2} \PY{o}{*} \PY{n}{x}\PY{p}{)}\PY{p}{)}
\end{Verbatim}


\begin{Verbatim}[commandchars=\\\{\}]
{\color{outcolor}Out[{\color{outcolor}4}]:} [<matplotlib.lines.Line2D at 0xa508b00>,
         <matplotlib.lines.Line2D at 0xa508d30>]
\end{Verbatim}
            
    \begin{center}
    \adjustimage{max size={0.9\linewidth}{0.9\paperheight}}{output_9_1.png}
    \end{center}
    { \hspace*{\fill} \\}
    
    使用字符串,给定线条参数:

    \begin{Verbatim}[commandchars=\\\{\}]
{\color{incolor}In [{\color{incolor}5}]:} \PY{n}{plot}\PY{p}{(}\PY{n}{x}\PY{p}{,} \PY{n}{sin}\PY{p}{(}\PY{n}{x}\PY{p}{)}\PY{p}{,} \PY{l+s+s1}{\PYZsq{}}\PY{l+s+s1}{r\PYZhy{}\PYZca{}}\PY{l+s+s1}{\PYZsq{}}\PY{p}{)}
\end{Verbatim}


\begin{Verbatim}[commandchars=\\\{\}]
{\color{outcolor}Out[{\color{outcolor}5}]:} [<matplotlib.lines.Line2D at 0xba6ea20>]
\end{Verbatim}
            
    \begin{center}
    \adjustimage{max size={0.9\linewidth}{0.9\paperheight}}{output_11_1.png}
    \end{center}
    { \hspace*{\fill} \\}
    
    多线条:

    \begin{Verbatim}[commandchars=\\\{\}]
{\color{incolor}In [{\color{incolor}9}]:} \PY{n}{plot}\PY{p}{(}\PY{n}{x}\PY{p}{,} \PY{n}{sin}\PY{p}{(}\PY{n}{x}\PY{p}{)}\PY{p}{,} \PY{l+s+s1}{\PYZsq{}}\PY{l+s+s1}{b\PYZhy{}o}\PY{l+s+s1}{\PYZsq{}}\PY{p}{,}
            \PY{n}{x}\PY{p}{,} \PY{n}{sin}\PY{p}{(}\PY{l+m+mi}{2} \PY{o}{*} \PY{n}{x}\PY{p}{)}\PY{p}{,} \PY{l+s+s1}{\PYZsq{}}\PY{l+s+s1}{r\PYZhy{}\PYZca{}}\PY{l+s+s1}{\PYZsq{}}\PY{p}{)}
\end{Verbatim}


\begin{Verbatim}[commandchars=\\\{\}]
{\color{outcolor}Out[{\color{outcolor}9}]:} [<matplotlib.lines.Line2D at 0xbcf1710>,
         <matplotlib.lines.Line2D at 0xbcf1940>]
\end{Verbatim}
            
    \begin{center}
    \adjustimage{max size={0.9\linewidth}{0.9\paperheight}}{output_13_1.png}
    \end{center}
    { \hspace*{\fill} \\}
    
    更多参数设置,请查阅帮助。事实上,字符串使用的格式与\textbf{Matlab}相同。

    \subsection{scatter 散点图}\label{scatter-ux6563ux70b9ux56fe}

    \begin{Shaded}
\begin{Highlighting}[]
\NormalTok{scatter(x, y)}
\NormalTok{scatter(x, y, size)}
\NormalTok{scatter(x, y, size, color)}
\end{Highlighting}
\end{Shaded}

假设我们想画二维散点图:

    \begin{Verbatim}[commandchars=\\\{\}]
{\color{incolor}In [{\color{incolor}10}]:} \PY{n}{plot}\PY{p}{(}\PY{n}{x}\PY{p}{,} \PY{n}{sin}\PY{p}{(}\PY{n}{x}\PY{p}{)}\PY{p}{,} \PY{l+s+s1}{\PYZsq{}}\PY{l+s+s1}{bo}\PY{l+s+s1}{\PYZsq{}}\PY{p}{)}
\end{Verbatim}


\begin{Verbatim}[commandchars=\\\{\}]
{\color{outcolor}Out[{\color{outcolor}10}]:} [<matplotlib.lines.Line2D at 0xbd6c0b8>]
\end{Verbatim}
            
    \begin{center}
    \adjustimage{max size={0.9\linewidth}{0.9\paperheight}}{output_17_1.png}
    \end{center}
    { \hspace*{\fill} \\}
    
    可以使用 \texttt{scatter} 达到同样的效果:

    \begin{Verbatim}[commandchars=\\\{\}]
{\color{incolor}In [{\color{incolor}11}]:} \PY{n}{scatter}\PY{p}{(}\PY{n}{x}\PY{p}{,} \PY{n}{sin}\PY{p}{(}\PY{n}{x}\PY{p}{)}\PY{p}{)}
\end{Verbatim}


\begin{Verbatim}[commandchars=\\\{\}]
{\color{outcolor}Out[{\color{outcolor}11}]:} <matplotlib.collections.PathCollection at 0xbd996d8>
\end{Verbatim}
            
    \begin{center}
    \adjustimage{max size={0.9\linewidth}{0.9\paperheight}}{output_19_1.png}
    \end{center}
    { \hspace*{\fill} \\}
    
    事实上,scatter函数与\textbf{Matlab}的用法相同,还可以指定它的大小,颜色等参数:

    \begin{Verbatim}[commandchars=\\\{\}]
{\color{incolor}In [{\color{incolor}12}]:} \PY{n}{x} \PY{o}{=} \PY{n}{rand}\PY{p}{(}\PY{l+m+mi}{200}\PY{p}{)}
         \PY{n}{y} \PY{o}{=} \PY{n}{rand}\PY{p}{(}\PY{l+m+mi}{200}\PY{p}{)}
         \PY{n}{size} \PY{o}{=} \PY{n}{rand}\PY{p}{(}\PY{l+m+mi}{200}\PY{p}{)} \PY{o}{*} \PY{l+m+mi}{30}
         \PY{n}{color} \PY{o}{=} \PY{n}{rand}\PY{p}{(}\PY{l+m+mi}{200}\PY{p}{)}
         \PY{n}{scatter}\PY{p}{(}\PY{n}{x}\PY{p}{,} \PY{n}{y}\PY{p}{,} \PY{n}{size}\PY{p}{,} \PY{n}{color}\PY{p}{)}
         \PY{c+c1}{\PYZsh{} 显示颜色条}
         \PY{n}{colorbar}\PY{p}{(}\PY{p}{)}
\end{Verbatim}


\begin{Verbatim}[commandchars=\\\{\}]
{\color{outcolor}Out[{\color{outcolor}12}]:} <matplotlib.colorbar.Colorbar instance at 0x000000000C31F448>
\end{Verbatim}
            
    \begin{center}
    \adjustimage{max size={0.9\linewidth}{0.9\paperheight}}{output_21_1.png}
    \end{center}
    { \hspace*{\fill} \\}
    
    \subsection{多图}\label{ux591aux56fe}

    使用figure()命令产生新的图像:

    \begin{Verbatim}[commandchars=\\\{\}]
{\color{incolor}In [{\color{incolor}13}]:} \PY{n}{t} \PY{o}{=} \PY{n}{linspace}\PY{p}{(}\PY{l+m+mi}{0}\PY{p}{,} \PY{l+m+mi}{2}\PY{o}{*}\PY{n}{pi}\PY{p}{,} \PY{l+m+mi}{50}\PY{p}{)}
         \PY{n}{x} \PY{o}{=} \PY{n}{sin}\PY{p}{(}\PY{n}{t}\PY{p}{)}
         \PY{n}{y} \PY{o}{=} \PY{n}{cos}\PY{p}{(}\PY{n}{t}\PY{p}{)}
         \PY{n}{figure}\PY{p}{(}\PY{p}{)}
         \PY{n}{plot}\PY{p}{(}\PY{n}{x}\PY{p}{)}
         \PY{n}{figure}\PY{p}{(}\PY{p}{)}
         \PY{n}{plot}\PY{p}{(}\PY{n}{y}\PY{p}{)}
\end{Verbatim}


\begin{Verbatim}[commandchars=\\\{\}]
{\color{outcolor}Out[{\color{outcolor}13}]:} [<matplotlib.lines.Line2D at 0xc680cf8>]
\end{Verbatim}
            
    \begin{center}
    \adjustimage{max size={0.9\linewidth}{0.9\paperheight}}{output_24_1.png}
    \end{center}
    { \hspace*{\fill} \\}
    
    \begin{center}
    \adjustimage{max size={0.9\linewidth}{0.9\paperheight}}{output_24_2.png}
    \end{center}
    { \hspace*{\fill} \\}
    
    或者使用 \texttt{subplot} 在一幅图中画多幅子图:

\begin{verbatim}
subplot(row, column, index)
\end{verbatim}

    \begin{Verbatim}[commandchars=\\\{\}]
{\color{incolor}In [{\color{incolor}15}]:} \PY{n}{subplot}\PY{p}{(}\PY{l+m+mi}{1}\PY{p}{,} \PY{l+m+mi}{2}\PY{p}{,} \PY{l+m+mi}{1}\PY{p}{)}
         \PY{n}{plot}\PY{p}{(}\PY{n}{x}\PY{p}{)}
         \PY{n}{subplot}\PY{p}{(}\PY{l+m+mi}{1}\PY{p}{,} \PY{l+m+mi}{2}\PY{p}{,} \PY{l+m+mi}{2}\PY{p}{)}
         \PY{n}{plot}\PY{p}{(}\PY{n}{y}\PY{p}{)}
\end{Verbatim}


\begin{Verbatim}[commandchars=\\\{\}]
{\color{outcolor}Out[{\color{outcolor}15}]:} [<matplotlib.lines.Line2D at 0xcd47518>]
\end{Verbatim}
            
    \begin{center}
    \adjustimage{max size={0.9\linewidth}{0.9\paperheight}}{output_26_1.png}
    \end{center}
    { \hspace*{\fill} \\}
    
    \subsection{向图中添加数据}\label{ux5411ux56feux4e2dux6dfbux52a0ux6570ux636e}

    默认多次 \texttt{plot} 会叠加:

    \begin{Verbatim}[commandchars=\\\{\}]
{\color{incolor}In [{\color{incolor}16}]:} \PY{n}{plot}\PY{p}{(}\PY{n}{x}\PY{p}{)}
         \PY{n}{plot}\PY{p}{(}\PY{n}{y}\PY{p}{)}
\end{Verbatim}


\begin{Verbatim}[commandchars=\\\{\}]
{\color{outcolor}Out[{\color{outcolor}16}]:} [<matplotlib.lines.Line2D at 0xcbcfd30>]
\end{Verbatim}
            
    \begin{center}
    \adjustimage{max size={0.9\linewidth}{0.9\paperheight}}{output_29_1.png}
    \end{center}
    { \hspace*{\fill} \\}
    
    可以跟\textbf{Matlab}类似用 hold(False)关掉,这样新图会将原图覆盖:

    \begin{Verbatim}[commandchars=\\\{\}]
{\color{incolor}In [{\color{incolor}17}]:} \PY{n}{plot}\PY{p}{(}\PY{n}{x}\PY{p}{)}
         \PY{n}{hold}\PY{p}{(}\PY{k+kc}{False}\PY{p}{)}
         \PY{n}{plot}\PY{p}{(}\PY{n}{y}\PY{p}{)}
         \PY{c+c1}{\PYZsh{} 恢复原来设定}
         \PY{n}{hold}\PY{p}{(}\PY{k+kc}{True}\PY{p}{)}
\end{Verbatim}


\begin{Verbatim}[commandchars=\\\{\}]
{\color{outcolor}Out[{\color{outcolor}17}]:} [<matplotlib.lines.Line2D at 0xcf4b9b0>]
\end{Verbatim}
            
    \begin{center}
    \adjustimage{max size={0.9\linewidth}{0.9\paperheight}}{output_31_1.png}
    \end{center}
    { \hspace*{\fill} \\}
    
    \subsection{标签}\label{ux6807ux7b7e}

    可以在 \texttt{plot} 中加入 \texttt{label} ,使用 \texttt{legend}
加上图例:

    \begin{Verbatim}[commandchars=\\\{\}]
{\color{incolor}In [{\color{incolor}19}]:} \PY{n}{plot}\PY{p}{(}\PY{n}{x}\PY{p}{,} \PY{n}{label}\PY{o}{=}\PY{l+s+s1}{\PYZsq{}}\PY{l+s+s1}{sin}\PY{l+s+s1}{\PYZsq{}}\PY{p}{)}
         \PY{n}{plot}\PY{p}{(}\PY{n}{y}\PY{p}{,} \PY{n}{label}\PY{o}{=}\PY{l+s+s1}{\PYZsq{}}\PY{l+s+s1}{cos}\PY{l+s+s1}{\PYZsq{}}\PY{p}{)}
         \PY{n}{legend}\PY{p}{(}\PY{p}{)}
\end{Verbatim}


\begin{Verbatim}[commandchars=\\\{\}]
{\color{outcolor}Out[{\color{outcolor}19}]:} <matplotlib.legend.Legend at 0xd2089b0>
\end{Verbatim}
            
    \begin{center}
    \adjustimage{max size={0.9\linewidth}{0.9\paperheight}}{output_34_1.png}
    \end{center}
    { \hspace*{\fill} \\}
    
    或者直接在 \texttt{legend}中加入:

    \begin{Verbatim}[commandchars=\\\{\}]
{\color{incolor}In [{\color{incolor}21}]:} \PY{n}{plot}\PY{p}{(}\PY{n}{x}\PY{p}{)}
         \PY{n}{plot}\PY{p}{(}\PY{n}{y}\PY{p}{)}
         \PY{n}{legend}\PY{p}{(}\PY{p}{[}\PY{l+s+s1}{\PYZsq{}}\PY{l+s+s1}{sin}\PY{l+s+s1}{\PYZsq{}}\PY{p}{,} \PY{l+s+s1}{\PYZsq{}}\PY{l+s+s1}{cos}\PY{l+s+s1}{\PYZsq{}}\PY{p}{]}\PY{p}{)}
\end{Verbatim}


\begin{Verbatim}[commandchars=\\\{\}]
{\color{outcolor}Out[{\color{outcolor}21}]:} <matplotlib.legend.Legend at 0xd51fb00>
\end{Verbatim}
            
    \begin{center}
    \adjustimage{max size={0.9\linewidth}{0.9\paperheight}}{output_36_1.png}
    \end{center}
    { \hspace*{\fill} \\}
    
    \subsection{坐标轴,标题,网格}\label{ux5750ux6807ux8f74ux6807ux9898ux7f51ux683c}

    可以设置坐标轴的标签和标题:

    \begin{Verbatim}[commandchars=\\\{\}]
{\color{incolor}In [{\color{incolor}22}]:} \PY{n}{plot}\PY{p}{(}\PY{n}{x}\PY{p}{,} \PY{n}{sin}\PY{p}{(}\PY{n}{x}\PY{p}{)}\PY{p}{)}
         \PY{n}{xlabel}\PY{p}{(}\PY{l+s+s1}{\PYZsq{}}\PY{l+s+s1}{radians}\PY{l+s+s1}{\PYZsq{}}\PY{p}{)}
         \PY{c+c1}{\PYZsh{} 可以设置字体大小}
         \PY{n}{ylabel}\PY{p}{(}\PY{l+s+s1}{\PYZsq{}}\PY{l+s+s1}{amplitude}\PY{l+s+s1}{\PYZsq{}}\PY{p}{,} \PY{n}{fontsize}\PY{o}{=}\PY{l+s+s1}{\PYZsq{}}\PY{l+s+s1}{large}\PY{l+s+s1}{\PYZsq{}}\PY{p}{)}
         \PY{n}{title}\PY{p}{(}\PY{l+s+s1}{\PYZsq{}}\PY{l+s+s1}{Sin(x)}\PY{l+s+s1}{\PYZsq{}}\PY{p}{)}
\end{Verbatim}


\begin{Verbatim}[commandchars=\\\{\}]
{\color{outcolor}Out[{\color{outcolor}22}]:} <matplotlib.text.Text at 0xd727dd8>
\end{Verbatim}
            
    \begin{center}
    \adjustimage{max size={0.9\linewidth}{0.9\paperheight}}{output_39_1.png}
    \end{center}
    { \hspace*{\fill} \\}
    
    用 'grid()' 来显示网格:

    \begin{Verbatim}[commandchars=\\\{\}]
{\color{incolor}In [{\color{incolor}23}]:} \PY{n}{plot}\PY{p}{(}\PY{n}{x}\PY{p}{,} \PY{n}{sin}\PY{p}{(}\PY{n}{x}\PY{p}{)}\PY{p}{)}
         \PY{n}{xlabel}\PY{p}{(}\PY{l+s+s1}{\PYZsq{}}\PY{l+s+s1}{radians}\PY{l+s+s1}{\PYZsq{}}\PY{p}{)}
         \PY{n}{ylabel}\PY{p}{(}\PY{l+s+s1}{\PYZsq{}}\PY{l+s+s1}{amplitude}\PY{l+s+s1}{\PYZsq{}}\PY{p}{,} \PY{n}{fontsize}\PY{o}{=}\PY{l+s+s1}{\PYZsq{}}\PY{l+s+s1}{large}\PY{l+s+s1}{\PYZsq{}}\PY{p}{)}
         \PY{n}{title}\PY{p}{(}\PY{l+s+s1}{\PYZsq{}}\PY{l+s+s1}{Sin(x)}\PY{l+s+s1}{\PYZsq{}}\PY{p}{)}
         \PY{n}{grid}\PY{p}{(}\PY{p}{)}
\end{Verbatim}


    \begin{center}
    \adjustimage{max size={0.9\linewidth}{0.9\paperheight}}{output_41_0.png}
    \end{center}
    { \hspace*{\fill} \\}
    
    \subsection{清除、关闭图像}\label{ux6e05ux9664ux5173ux95edux56feux50cf}

    清除已有的图像使用:

\begin{verbatim}
clf()
\end{verbatim}

关闭当前图像:

\begin{verbatim}
close()
\end{verbatim}

关闭所有图像:

\begin{verbatim}
close('all')
\end{verbatim}

    \subsection{imshow 显示图片}\label{imshow-ux663eux793aux56feux7247}

    灰度图片可以看成二维数组:

    \begin{Verbatim}[commandchars=\\\{\}]
{\color{incolor}In [{\color{incolor}25}]:} \PY{c+c1}{\PYZsh{} 导入lena图片}
         \PY{k+kn}{from} \PY{n+nn}{scipy}\PY{n+nn}{.}\PY{n+nn}{misc} \PY{k}{import} \PY{n}{lena}
         \PY{n}{img} \PY{o}{=} \PY{n}{lena}\PY{p}{(}\PY{p}{)}
         \PY{n}{img}
\end{Verbatim}


\begin{Verbatim}[commandchars=\\\{\}]
{\color{outcolor}Out[{\color{outcolor}25}]:} array([[162, 162, 162, {\ldots}, 170, 155, 128],
                [162, 162, 162, {\ldots}, 170, 155, 128],
                [162, 162, 162, {\ldots}, 170, 155, 128],
                {\ldots}, 
                [ 43,  43,  50, {\ldots}, 104, 100,  98],
                [ 44,  44,  55, {\ldots}, 104, 105, 108],
                [ 44,  44,  55, {\ldots}, 104, 105, 108]])
\end{Verbatim}
            
    我们可以用 \texttt{imshow()} 来显示图片数据:

    \begin{Verbatim}[commandchars=\\\{\}]
{\color{incolor}In [{\color{incolor}26}]:} \PY{n}{imshow}\PY{p}{(}\PY{n}{img}\PY{p}{,}
                \PY{c+c1}{\PYZsh{} 设置坐标范围}
               \PY{n}{extent} \PY{o}{=} \PY{p}{[}\PY{o}{\PYZhy{}}\PY{l+m+mi}{25}\PY{p}{,} \PY{l+m+mi}{25}\PY{p}{,} \PY{o}{\PYZhy{}}\PY{l+m+mi}{25}\PY{p}{,} \PY{l+m+mi}{25}\PY{p}{]}\PY{p}{,}
                \PY{c+c1}{\PYZsh{} 设置colormap}
               \PY{n}{cmap} \PY{o}{=} \PY{n}{cm}\PY{o}{.}\PY{n}{bone}\PY{p}{)}
         \PY{n}{colorbar}\PY{p}{(}\PY{p}{)}
\end{Verbatim}


\begin{Verbatim}[commandchars=\\\{\}]
{\color{outcolor}Out[{\color{outcolor}26}]:} <matplotlib.colorbar.Colorbar instance at 0x000000000DECFD88>
\end{Verbatim}
            
    \begin{center}
    \adjustimage{max size={0.9\linewidth}{0.9\paperheight}}{output_48_1.png}
    \end{center}
    { \hspace*{\fill} \\}
    
    更多参数和用法可以参阅帮助。

    这里 \texttt{cm} 表示 \texttt{colormap},可以看它的种类:

    \begin{Verbatim}[commandchars=\\\{\}]
{\color{incolor}In [{\color{incolor}28}]:} \PY{n+nb}{dir}\PY{p}{(}\PY{n}{cm}\PY{p}{)}
\end{Verbatim}


\begin{Verbatim}[commandchars=\\\{\}]
{\color{outcolor}Out[{\color{outcolor}28}]:} [u'Accent',
          u'Accent\_r',
          u'Blues',
          u'Blues\_r',
          u'BrBG',
          u'BrBG\_r',
          u'BuGn',
          u'BuGn\_r',
          u'BuPu',
          u'BuPu\_r',
          u'CMRmap',
          u'CMRmap\_r',
          u'Dark2',
          u'Dark2\_r',
          u'GnBu',
          u'GnBu\_r',
          u'Greens',
          u'Greens\_r',
          u'Greys',
          u'Greys\_r',
          'LUTSIZE',
          u'OrRd',
          u'OrRd\_r',
          u'Oranges',
          u'Oranges\_r',
          u'PRGn',
          u'PRGn\_r',
          u'Paired',
          u'Paired\_r',
          u'Pastel1',
          u'Pastel1\_r',
          u'Pastel2',
          u'Pastel2\_r',
          u'PiYG',
          u'PiYG\_r',
          u'PuBu',
          u'PuBuGn',
          u'PuBuGn\_r',
          u'PuBu\_r',
          u'PuOr',
          u'PuOr\_r',
          u'PuRd',
          u'PuRd\_r',
          u'Purples',
          u'Purples\_r',
          u'RdBu',
          u'RdBu\_r',
          u'RdGy',
          u'RdGy\_r',
          u'RdPu',
          u'RdPu\_r',
          u'RdYlBu',
          u'RdYlBu\_r',
          u'RdYlGn',
          u'RdYlGn\_r',
          u'Reds',
          u'Reds\_r',
          'ScalarMappable',
          u'Set1',
          u'Set1\_r',
          u'Set2',
          u'Set2\_r',
          u'Set3',
          u'Set3\_r',
          u'Spectral',
          u'Spectral\_r',
          u'Wistia',
          u'Wistia\_r',
          u'YlGn',
          u'YlGnBu',
          u'YlGnBu\_r',
          u'YlGn\_r',
          u'YlOrBr',
          u'YlOrBr\_r',
          u'YlOrRd',
          u'YlOrRd\_r',
          '\_\_builtins\_\_',
          '\_\_doc\_\_',
          '\_\_file\_\_',
          '\_\_name\_\_',
          '\_\_package\_\_',
          '\_generate\_cmap',
          '\_reverse\_cmap\_spec',
          '\_reverser',
          'absolute\_import',
          u'afmhot',
          u'afmhot\_r',
          u'autumn',
          u'autumn\_r',
          u'binary',
          u'binary\_r',
          u'bone',
          u'bone\_r',
          u'brg',
          u'brg\_r',
          u'bwr',
          u'bwr\_r',
          'cbook',
          'cmap\_d',
          'cmapname',
          'colors',
          u'cool',
          u'cool\_r',
          u'coolwarm',
          u'coolwarm\_r',
          u'copper',
          u'copper\_r',
          'cubehelix',
          u'cubehelix\_r',
          'datad',
          'division',
          u'flag',
          u'flag\_r',
          'get\_cmap',
          u'gist\_earth',
          u'gist\_earth\_r',
          u'gist\_gray',
          u'gist\_gray\_r',
          u'gist\_heat',
          u'gist\_heat\_r',
          u'gist\_ncar',
          u'gist\_ncar\_r',
          u'gist\_rainbow',
          u'gist\_rainbow\_r',
          u'gist\_stern',
          u'gist\_stern\_r',
          u'gist\_yarg',
          u'gist\_yarg\_r',
          u'gnuplot',
          u'gnuplot2',
          u'gnuplot2\_r',
          u'gnuplot\_r',
          u'gray',
          u'gray\_r',
          u'hot',
          u'hot\_r',
          u'hsv',
          u'hsv\_r',
          u'jet',
          u'jet\_r',
          'ma',
          'mpl',
          u'nipy\_spectral',
          u'nipy\_spectral\_r',
          'np',
          u'ocean',
          u'ocean\_r',
          'os',
          u'pink',
          u'pink\_r',
          'print\_function',
          u'prism',
          u'prism\_r',
          u'rainbow',
          u'rainbow\_r',
          'register\_cmap',
          'revcmap',
          u'seismic',
          u'seismic\_r',
          'six',
          'spec',
          'spec\_reversed',
          u'spectral',
          u'spectral\_r',
          u'spring',
          u'spring\_r',
          u'summer',
          u'summer\_r',
          u'terrain',
          u'terrain\_r',
          'unicode\_literals',
          u'winter',
          u'winter\_r']
\end{Verbatim}
            
    使用不同的 \texttt{colormap} 会有不同的显示效果。

    \begin{Verbatim}[commandchars=\\\{\}]
{\color{incolor}In [{\color{incolor}29}]:} \PY{n}{imshow}\PY{p}{(}\PY{n}{img}\PY{p}{,} \PY{n}{cmap}\PY{o}{=}\PY{n}{cm}\PY{o}{.}\PY{n}{RdGy\PYZus{}r}\PY{p}{)}
\end{Verbatim}


\begin{Verbatim}[commandchars=\\\{\}]
{\color{outcolor}Out[{\color{outcolor}29}]:} <matplotlib.image.AxesImage at 0xe0883c8>
\end{Verbatim}
            
    \begin{center}
    \adjustimage{max size={0.9\linewidth}{0.9\paperheight}}{output_53_1.png}
    \end{center}
    { \hspace*{\fill} \\}
    
    \subsection{从脚本中运行}\label{ux4eceux811aux672cux4e2dux8fd0ux884c}

    在脚本中使用 \texttt{plot} 时,通常图像是不会直接显示的,需要增加
\texttt{show()} 选项,只有在遇到 \texttt{show()}
命令之后,图像才会显示。

    \subsection{直方图}\label{ux76f4ux65b9ux56fe}

    从高斯分布随机生成1000个点得到的直方图:

    \begin{Verbatim}[commandchars=\\\{\}]
{\color{incolor}In [{\color{incolor}30}]:} \PY{n}{hist}\PY{p}{(}\PY{n}{randn}\PY{p}{(}\PY{l+m+mi}{1000}\PY{p}{)}\PY{p}{)}
\end{Verbatim}


\begin{Verbatim}[commandchars=\\\{\}]
{\color{outcolor}Out[{\color{outcolor}30}]:} (array([   2.,    7.,   37.,  119.,  216.,  270.,  223.,   82.,   31.,   13.]),
          array([-3.65594649, -2.98847032, -2.32099415, -1.65351798, -0.98604181,
                 -0.31856564,  0.34891053,  1.0163867 ,  1.68386287,  2.35133904,
                  3.01881521]),
          <a list of 10 Patch objects>)
\end{Verbatim}
            
    \begin{center}
    \adjustimage{max size={0.9\linewidth}{0.9\paperheight}}{output_58_1.png}
    \end{center}
    { \hspace*{\fill} \\}
    
    更多例子请参考下列网站:

http://matplotlib.org/gallery.html


    % Add a bibliography block to the postdoc
    
    
    
    \end{document}
